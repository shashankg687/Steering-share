\documentclass[12pt,showpacs,bibnotes,prl,onecolumn]{revtex4}
%\documentclass[12pt]{article}
%\documentclass[twocolumn,showpacs,twoside,superscriptaddress]{revtex4}
\usepackage{amssymb, amsbsy, amsmath, latexsym, dsfont, array, layout, graphicx,bm}
\usepackage{color,soul}

\newcommand{\fig}[1]{Fig.~\ref{#1}}
\newcommand{\tab}[1]{Table~\ref{#1}}
\newcommand{\one}{\mathds{1}}
\newcommand{\ket}[1]{\left|{#1}\right\rangle}
\newcommand{\bra}[1]{\left\langle{#1}\right|}
\newcommand{\braket}[2]{\langle{#1}|{#2}\rangle}
\newcommand{\ketbrad}[1]{\left|{#1}\rangle\!\langle{#1}\right|}
\newcommand{\ketbra}[2]{\left|{#1}\rangle\!\langle{#2}\right|}

\newcommand{\red}{\color[rgb]{0.8,0,0}}
\newcommand{\blue}{\color[rgb]{0,0,0.6}}
\newcommand{\green}{\color[rgb]{0.0,0.7,0.0}}


% commenting commands
\usepackage[normalem]{ulem}%This enables strike through text using the \sout{} command.
\usepackage{color}
\newcommand{\Tin}[1]{{\color{red} #1}}
\newcommand{\Tout}[1]{{\color{blue} \sout{#1}}}

\begin{document}

\begin{center}
{\bf {Responses to the reports on the manuscript AU12015A Gupta: ``Genuine Einstein-Podolsky-Rosen steering of three-qubit states by
    multiple sequential observers"}}
\end{center}



\begin{center}
  \textbf{Reply to the Second Report of the First Referee}
\end{center}

We thank  the referee for going through the revised manuscript and our response to his/her previous report. In the following, we reproduce the referee’s report verbatim and address the referee’s comments.

{\red{\bf{Referee's Comment:}}} \textit{I believe the authors have successfully addressed my concerns, and I
recommend publication to Physical Review A as a regular article.}

{\blue{\bf{Author's Response:}}}  We appreciate that referee's concerns have been successfully addressed and thank him/her for recommending our manuscript for publication.

\clearpage
\begin{center}
  \textbf{Reply to the Second Report of the Second Referee}
\end{center}

We thank  the referee for going through the revised manuscript and our response to his/her previous report. In the following, we reproduce the referee’s report verbatim and address the referee’s comments.

{\red{\bf{Referee's Comment:}}} \textit{The revised version of the manuscript addressed my main concern. I can
therefore recommend publication.}

{\blue{\bf{Author's Response:}}}  We are delighted that we have addressed the main concern of the referee and thank him/her for recommending our manuscript for publication.

\clearpage
\begin{center}
  \textbf{Reply to the Report of the Third Referee}
\end{center}

We thank the referee for his/her report, and also for the criticisms which have provided us an opportunity to improve the presentation. In the following, we reproduce the report verbatim and address all the comments and concerns, providing details of the changes made in the revised version.

{\red{\bf{Referee's Comment:}}} \textit{I have taken a look at the revised paper and the previous
communications between the authors and the referees. Both referees
have indicated that the present work is very close in style and
substance to previous work involving the authors. This might be
problematic since the authors formulate a rather complicated scenario
and present numerical results for a specific method of evaluation. The
question is whether the data obtained in the theoretical analysis is
of sufficient interest to justify publication in Physical Review A.}

{\blue{\bf{Author's Response:}}}  We acknowledge that our group has done many interesting work in this direction previously. The scenarios that we have considered in the present work are fundamentally different from the previous works and are key to illustrate multipartite steering phenomenon and develop multipartite secret sharing and cryptography. Further relevance and significance of the present paper has been discussed in appropriate places in Sections I and IV of the last submitted version of the manuscript. \\



{\red{\bf{Referee's Comment:}}} \textit{After carefully reading the paper and giving the contents some
thought, I have come to the conclusion that the paper makes a valid
technical contribution to the field. Although I would personally
prefer an approach that reveals more about the underlying physics
responsible for the non-local statistics investigated in the paper, I
admit that a detailed analysis of a specific case might be helpful as
an accessible illustration of the mechanism of steering in a
multi-partite context.}

{\blue{\bf{Author's Response:}}}  We thank the referee for carefully reviewing the manuscript. We appreciate that the referee found our work technically sound and helpful in illustrating multipartite steering. \\

{\red{\bf{Referee's Comment:}}} \textit{I am a bit concerned about the extreme reliance of the present work on
results established elsewhere. In particular, it is not entirely clear
to me whether the steering inequalities introduced in Eqs. (6) and (7)
are really the most representative or the most useful types of
steering inequalities for multi-partite systems. My impression is that
they are chosen mainly because they fit the method used to analyze the
effects of sequential measurements. Fortunately, the description of
the steering scenario given in the paper is sufficiently clear to
identify the particular steering effect identified by the
inequalities. In the end, the merit of the paper is in the
presentation of specific results, and some choices need to be made
when doing so. If the authors have a good explanation why they chose
the specific steering criteria given in Eqs. (6) and (7) it would be
helpful to add it in the text. However, I understand that it is
difficult to add to the present paper without losing some of the
clarity that would otherwise be its strong point.
}

{\blue{\bf{Author's Response:}}}  We have taken these inequalities (Equations: 6, 7, 8, 9) as tools to detect genuine tripartite steering because such inequalities detect the genuine steering solely from the measurement correlations in the steering scenario, thus acting as experimentally testable genuine steering witnesses. However, one can choose other types of genuine tripartite EPR steering witnesses as well in the context of the present study. In view of the referee's suggestion, we have added few sentences in the last paragraph of column-I, page-4, to clarify further the relevance of our choice of the specific steering criteria used in the present context. \\

{\red{\bf{Referee's Comment:}}} \textit{In summary, the authors provide a lot of detail regarding the specific
steering protocol they have chosen to investigate. The results are
useful as an illustration of the actual steering process in a
multi-partite setting with sequential measurements. The technical
quality of the analysis is good and the presentation is clear. I
therefore think that the paper is suitable for publication in Physical
Review A.}

{\blue{\bf{Author's Response:}}} We appreciate that the referee liked the technical analysis and relevance of our work in illustrating multipartite steering process. We thank the referee for recommending our work for publication.\\



\clearpage
\begin{center}
   \textbf{List of changes}
\end{center}

All the major modifications/additions incorporated in the revised manuscript are listed below.
\begin{itemize}
\item The affiliation of one of the author (Debarshi Das) has been updated.
\item Section IIA, page 4, column 1: Few sentences are added in the last paragraph to clarify the relevance of our choice of the specific steering criteria used in the present context.
\item Caption of tables and figures has been left aligned for better clarity.
\item Acknowledgement section has been modified slightly.


\end{itemize}




\end{document} 